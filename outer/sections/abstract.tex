Large language models (LLMs) are rapidly improving in their ability to generate
text that closely resembles human writing. This has led to a growing interest
in using LLMs to assist in academic writing, with the potential to improve the
quality and efficiency of the writing process. Most academic journals now have
policies in place to govern the use of LLMs in academic writing, and most 
isallow the citation of LLMs as authors or co-authors. In this paper, we explore
the current ability for an LLM to function as a co-author in academic writing,
and in particular, whether or not an LLM can produce novel ideas and insights to
an academic paper. We present a case study in which we use the Llama-2-7b model
to assist in writing a paper connecting algorithmic fairness to Nozick's title
of entitlement justice, in contrast to a large existing body of research
connecting algorithmic fairness to egalitarian justice. We perform a qualitative
analysis of the LLM's ability to contribute to the paper. We find that the LLM
is able to recommend and summarize relevant literature, generate text that is
coherent, well-structured, and relevant to the topic, and provide novel insights
and ideas that are not present in the literature, but cannot produce or defend a
full academic argument.