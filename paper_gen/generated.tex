\section{Entitlement Theory and Algorithmic Fairness: A Nozickian
Framework for Just Decision-Making
Systems}\label{entitlement-theory-and-algorithmic-fairness-a-nozickian-framework-for-just-decision-making-systems}

\subsection{Abstract}\label{abstract}

This thesis examines the theoretical underpinnings of algorithmic
fairness measures through the lens of Nozick's libertarian theory of
entitlement justice. While current algorithmic fairness research heavily
draws from egalitarian theories of distributive justice, focusing on
equality of outcomes across demographic groups, this perspective has
neglected libertarian conceptions of fairness that emphasize procedural
justice and respect for individual rights. By developing a Nozickian
framework for algorithmic fairness, this thesis offers a novel
perspective on how algorithmic decision systems might be evaluated
against principles of acquisition, transfer, and rectification. The
thesis proposes a formal measure of ``procedural entitlement fairness''
and applies it to case studies in algorithmic lending and hiring
decisions. This approach illuminates the philosophical tensions inherent
in algorithmic governance and suggests that combining egalitarian and
libertarian perspectives may yield more nuanced fairness metrics that
respect both group equality and individual procedural rights. The
framework developed here demonstrates that Nozick's theories, though
rarely applied to algorithmic contexts, offer valuable insights for
addressing the normative challenges posed by automated decision-making
systems.

\subsection{1. Introduction}\label{introduction}

The increasing role of algorithms in decision-making processes that
affect human lives---from loan approvals and hiring decisions to
criminal sentencing and resource allocation---has prompted significant
ethical concerns about fairness and justice. As algorithms make or
inform consequential decisions, questions about what constitutes
``fair'' algorithmic behavior have become central to both technical and
philosophical discussions (Barocas et al., 2019). The field of
algorithmic fairness has emerged as a response to these concerns, with
researchers proposing various formal definitions and statistical metrics
to evaluate whether an algorithm treats different demographic groups
equitably.

These fairness metrics, such as demographic parity, equalized odds, and
equal opportunity, have predominantly been conceptualized through an
egalitarian lens that focuses on the distribution of outcomes across
different groups (Corbett-Davies \& Goel, 2018). This approach aligns
with dominant theories of distributive justice in the philosophical
literature, particularly those that emphasize equality of opportunity or
outcomes. However, this focus has led to a notable gap: the relative
absence of libertarian perspectives on justice, particularly Robert
Nozick's entitlement theory, from discussions of algorithmic fairness.

Nozick's theory of justice, as outlined in \emph{Anarchy, State, and
Utopia} (1974), presents a radically different conception of fairness
than most egalitarian theories. Rather than focusing on patterns of
distribution (such as equality), Nozick argues that justice is
historical and procedural---a distribution is just if it arose through
just processes of acquisition and transfer, regardless of the resulting
pattern. This perspective has been largely overlooked in algorithmic
fairness research, despite its significant influence in political
philosophy.

This theoretical gap raises several important questions: How might
algorithmic fairness be conceptualized from a Nozickian perspective?
What would a measure of algorithmic fairness based on entitlement theory
look like? How would such a measure compare to existing egalitarian
metrics? And what insights might a libertarian approach offer to the
broader discussion of algorithmic ethics?

\subsubsection{1.1 Research Objectives}\label{research-objectives}

This thesis aims to address these questions through the following
objectives:

\begin{enumerate}
\def\labelenumi{\arabic{enumi}.}
\item
  To elucidate the philosophical foundations of current algorithmic
  fairness metrics, highlighting their egalitarian assumptions and
  limitations.
\item
  To develop a theoretical framework for algorithmic fairness grounded
  in Nozick's entitlement theory of justice.
\item
  To formulate a formal measure or set of measures that can evaluate
  algorithmic fairness according to Nozickian principles.
\item
  To apply this framework to case studies in algorithmic
  decision-making, comparing the results with those derived from
  traditional egalitarian measures.
\item
  To explore the philosophical implications and practical challenges of
  implementing a libertarian approach to algorithmic fairness.
\end{enumerate}

\subsubsection{1.2 Significance and
Contribution}\label{significance-and-contribution}

This research contributes to both philosophical and technical
discussions of algorithmic fairness in several ways. First, it broadens
the philosophical foundations of algorithmic ethics by introducing a
perspective that has been underrepresented in the literature. Second, it
offers a novel conceptual framework for evaluating algorithmic systems
that complements existing approaches. Third, it provides practical
measures that can be used by developers and policymakers to assess
algorithms from a libertarian perspective. Finally, it illuminates the
broader philosophical tensions that underlie debates about fairness in
the digital age.

By bringing Nozick's entitlement theory into conversation with
algorithmic fairness, this thesis aims to enrich our understanding of
what justice means in the context of automated decision-making and to
provide new tools for ensuring that algorithms align with diverse
conceptions of fairness.

\subsubsection{1.3 Methodological
Approach}\label{methodological-approach}

This thesis employs a methodology that combines philosophical analysis
with formal modeling. The approach includes:

\begin{enumerate}
\def\labelenumi{\arabic{enumi}.}
\item
  Critical analysis of the philosophical foundations of existing
  algorithmic fairness metrics, with particular attention to their
  implicit ethical assumptions.
\item
  Exegetical analysis of Nozick's entitlement theory to extract core
  principles relevant to algorithmic decision-making.
\item
  Formal development of metrics that operationalize Nozickian principles
  in the context of algorithmic systems.
\item
  Application of these metrics to hypothetical and real-world case
  studies, with comparative analysis against traditional fairness
  measures.
\item
  Philosophical reflection on the implications, limitations, and
  potential extensions of the proposed framework.
\end{enumerate}

This methodology allows for a rigorous exploration of how libertarian
principles might be translated into the technical language of
algorithmic fairness while maintaining philosophical integrity.

\subsubsection{1.4 Thesis Structure}\label{thesis-structure}

The remainder of this thesis is structured as follows:

Section 2 provides background on algorithmic fairness, including an
overview of key metrics and their philosophical underpinnings, as well
as an introduction to theories of distributive justice with emphasis on
the distinctions between egalitarian and libertarian approaches.

Section 3 develops a Nozickian framework for algorithmic fairness,
articulating how the principles of just acquisition, transfer, and
rectification can be applied to algorithmic decision-making. This
section also proposes formal measures for evaluating algorithmic
fairness from a libertarian perspective.

Section 4 applies the developed framework to case studies in algorithmic
lending and hiring, comparing the results with those obtained using
traditional fairness metrics.

Section 5 discusses the implications of the Nozickian approach,
addressing potential objections and limitations, as well as suggesting
directions for future research.

Section 6 concludes by summarizing the key contributions of the thesis
and reflecting on the broader significance of incorporating diverse
philosophical perspectives into algorithmic ethics.

\subsection{2. Background}\label{background}

\subsubsection{2.1 Algorithmic Fairness: Concepts and
Metrics}\label{algorithmic-fairness-concepts-and-metrics}

\paragraph{2.1.1 The Rise of Algorithmic
Decision-Making}\label{the-rise-of-algorithmic-decision-making}

The proliferation of algorithmic decision-making systems across various
domains of social life represents one of the most significant
technological developments of the 21st century. These systems---ranging
from simple rule-based algorithms to complex machine learning
models---now influence or determine outcomes in areas such as financial
lending (Fuster et al., 2022), hiring (Bogen \& Rieke, 2018), criminal
justice (Angwin et al., 2016), healthcare resource allocation (Obermeyer
et al., 2019), and social service provision (Eubanks, 2018). The
increasing reliance on these systems stems from their perceived
benefits: efficiency, consistency, scale, and the potential to overcome
human biases and limitations.

However, the adoption of algorithmic decision-making has been
accompanied by growing concerns about fairness and discrimination.
Research has demonstrated that algorithms can reproduce or even amplify
existing social biases (Barocas \& Selbst, 2016; Noble, 2018). These
biases may emerge through various mechanisms, including biased training
data, problematic feature selection, proxy discrimination, and the
optimization of objectives that disadvantage certain groups (Hellman,
2020). The recognition of these issues has given rise to the field of
algorithmic fairness, which seeks to develop techniques for detecting,
measuring, and mitigating unfairness in algorithmic systems.

\paragraph{2.1.2 Formal Definitions of
Fairness}\label{formal-definitions-of-fairness}

The field of algorithmic fairness has produced numerous formal
definitions and statistical metrics to evaluate whether an algorithm
treats different demographic groups equitably. These definitions
generally fall into several categories, each reflecting different
intuitions about what constitutes fair treatment. The most prominent
categories include:

\textbf{Independence-based fairness (Demographic Parity)}: This
definition requires that the algorithm's decision be independent of
protected attributes such as race or gender. Formally, if R represents
the algorithm's decision (e.g., approve/deny) and A represents a
protected attribute (e.g., race), demographic parity requires that P(R =
r \textbar{} A = a) = P(R = r \textbar{} A = a') for all values r, a,
and a' (Dwork et al., 2012). In other words, the probability of each
outcome should be equal across all demographic groups.

\textbf{Separation-based fairness (Equalized Odds)}: This definition
requires that the algorithm's decision be independent of protected
attributes, conditional on the actual outcome Y. Formally, P(R = r
\textbar{} A = a, Y = y) = P(R = r \textbar{} A = a', Y = y) for all r,
a, a', and y (Hardt et al., 2016). This means that both true positive
rates and false positive rates should be equal across demographic
groups.

\textbf{Equal Opportunity}: A relaxation of equalized odds, this
definition only requires equality of true positive rates across groups:
P(R = 1 \textbar{} A = a, Y = 1) = P(R = 1 \textbar{} A = a', Y = 1) for
all a and a' (Hardt et al., 2016).

\textbf{Sufficiency-based fairness (Calibration)}: This definition
requires that the algorithm's predicted probability of an outcome match
the actual probability of that outcome within each group. Formally, P(Y
= 1 \textbar{} R = r, A = a) = P(Y = 1 \textbar{} R = r, A = a') = r for
all a and a' (Kleinberg et al., 2017).

\textbf{Individual Fairness}: Moving beyond group-level measures,
individual fairness requires that similar individuals receive similar
treatment. Formally, if \(d(x, x')\) represents the distance between
individuals x and x' in a relevant metric space, and \(d'(R(x), R(x'))\)
represents the distance between the algorithm's decisions for these
individuals, then \(d'(R(x), R(x')) \leq d(x, x')\)(Dwork et al., 2012).

These definitions represent different interpretations of fairness, often
stemming from different normative intuitions about what constitutes just
treatment. Importantly, research has demonstrated that many of these
definitions are mathematically incompatible---it is generally impossible
to satisfy all of them simultaneously, except in highly constrained or
trivial scenarios (Kleinberg et al., 2017; Chouldechova, 2017). This
incompatibility necessitates choices about which fairness criteria to
prioritize in any given context, choices that ultimately reflect value
judgments about the nature of fairness itself.

\paragraph{2.1.3 Limitations and Critiques of Current
Approaches}\label{limitations-and-critiques-of-current-approaches}

While formal fairness metrics have provided valuable tools for
identifying and addressing certain forms of algorithmic bias, they have
been subject to various critiques. These critiques can be broadly
categorized as follows:

\textbf{Technical Limitations}: As noted above, the impossibility
results of Kleinberg et al.~(2017) and Chouldechova (2017) demonstrate
that different formal fairness criteria cannot generally be satisfied
simultaneously. This means that choosing to satisfy one criterion
necessarily involves trade-offs with others. Additionally, many fairness
metrics are sensitive to how the problem is framed and which variables
are included in the model (Corbett-Davies \& Goel, 2018).

\textbf{Contextual Inadequacy}: Fairness metrics often fail to account
for the specific social, historical, and institutional contexts in which
algorithms operate. What constitutes fair treatment may vary across
different domains and social settings, making generic statistical
definitions insufficient (Green, 2018; Selbst et al., 2019).

\textbf{Procedural Neglect}: Most fairness metrics focus on the
distribution of outcomes rather than the processes by which those
outcomes are determined. This neglects procedural aspects of fairness,
such as transparency, explainability, and the ability to contest
decisions (Grgić-Hlača et al., 2018; Binns, 2018).

\textbf{Individualist vs.~Group-Based Tensions}: There is an inherent
tension between group-based fairness metrics, which aim to ensure
equality across demographic groups, and individualist approaches, which
emphasize treating similar individuals similarly regardless of group
membership (Binns, 2020; Dwork et al., 2012).

\textbf{Normative Ambiguity}: The choice of which fairness metric to use
involves implicit normative assumptions about what constitutes fair
treatment, yet these assumptions are often left unexamined. Different
metrics may align with different theories of justice, but this
connection is rarely made explicit (Fazelpour \& Lipton, 2020; Binns,
2018).

These limitations highlight the need for approaches to algorithmic
fairness that are philosophically grounded, contextually sensitive, and
capable of addressing both procedural and distributive aspects of
justice. This thesis aims to contribute to this effort by exploring how
Nozick's entitlement theory might offer a distinctive perspective on
algorithmic fairness---one that foregrounds procedural justice and
historical processes rather than distributive patterns.

\subsubsection{2.2 Theories of Distributive
Justice}\label{theories-of-distributive-justice}

\paragraph{2.2.1 Egalitarian Theories}\label{egalitarian-theories}

Egalitarian theories of distributive justice are primarily concerned
with the way goods, resources, opportunities, or welfare are distributed
across society. These theories generally hold that justice requires some
form of equality, though they differ in what exactly should be
equalized. The main variants of egalitarianism include:

\textbf{Strict Egalitarianism}: This view holds that justice requires
equality of outcomes---all individuals should receive equal shares of
goods or resources. This position is rarely defended in its pure form
due to various practical and theoretical objections (such as differences
in need, effort, or desert), but it serves as a baseline against which
other theories are often contrasted (Nielsen, 1979).

\textbf{Luck Egalitarianism}: This approach, associated with
philosophers such as Ronald Dworkin (2000), G.A. Cohen (1989), and
Richard Arneson (1989), aims to neutralize the effects of brute or
unchosen luck while holding individuals responsible for their choices.
Justice requires compensating individuals for disadvantages that result
from circumstances beyond their control (brute luck) but allows
inequalities that result from voluntary choices (option luck).

\textbf{Prioritarianism}: Rather than strict equality, prioritarianism
holds that justice requires giving priority to improving the condition
of those who are worst off. This view, associated with philosophers like
Derek Parfit (1997), differs from strict egalitarianism in that it is
concerned with absolute rather than relative levels of well-being.

\textbf{Rawlsian Justice as Fairness}: John Rawls's (1971) influential
theory combines egalitarian and prioritarian elements. His difference
principle permits inequalities only if they benefit the least advantaged
members of society. Additionally, his principle of fair equality of
opportunity requires that positions and offices be open to all under
conditions where individuals with similar talents and motivation have
similar prospects for success.

These egalitarian approaches share a focus on the pattern or structure
of distribution---they evaluate justice based on how goods or
opportunities are distributed at a given time, rather than the processes
by which that distribution came about. This pattern-based approach has
been influential in shaping algorithmic fairness metrics, particularly
those focused on ensuring equality of outcomes or opportunities across
demographic groups.

\paragraph{2.2.2 Libertarian Theories and Nozick's Entitlement
Theory}\label{libertarian-theories-and-nozicks-entitlement-theory}

In contrast to egalitarian theories, libertarian approaches to
distributive justice focus on individual rights and the processes by
which distributions arise, rather than the resulting patterns. The most
prominent libertarian theory of justice is Robert Nozick's entitlement
theory, articulated in his seminal work \emph{Anarchy, State, and
Utopia} (1974).

Nozick's theory is fundamentally historical and procedural---it holds
that a distribution is just if it arose through just processes,
regardless of the resulting pattern. The theory consists of three main
principles:

\textbf{The Principle of Justice in Acquisition}: This principle
concerns the original acquisition of holdings from an unowned state of
nature. Nozick draws on Locke's theory of property acquisition, which
holds that one can acquire previously unowned resources by mixing one's
labor with them, subject to the ``Lockean proviso'' that ``enough and as
good'' is left for others (Nozick, 1974, p.~175).

\textbf{The Principle of Justice in Transfer}: Once resources have been
justly acquired, they may be transferred to others. A transfer is just
if it occurs through voluntary exchange, gift, or bequest. Coerced
transfers, such as theft or fraud, violate this principle and result in
unjust holdings (Nozick, 1974, p.~150-153).

\textbf{The Principle of Rectification of Injustice}: This principle
addresses how to rectify past injustices in acquisition or transfer. If
current holdings resulted from unjust processes, rectification is
required to restore a just distribution. Nozick acknowledges the
complexity of determining what rectification would entail in real-world
scenarios with histories of injustice (Nozick, 1974, p.~152-153).

Nozick's theory explicitly rejects what he calls ``patterned'' theories
of justice---theories that specify that a distribution is to vary along
with some natural dimension, such as moral merit, need, or usefulness to
society. He argues that maintaining any pattern would require continuous
interference with people's ability to freely exchange what they own,
thus violating their rights (Nozick, 1974, p.~160-164). His famous
``Wilt Chamberlain'' example illustrates how liberty upsets patterns: if
we start with a just distribution (according to some pattern) and then
allow people to freely exchange their resources (e.g., by paying to
watch Wilt Chamberlain play basketball), the resulting distribution will
no longer conform to the original pattern. Nozick argues that preventing
this outcome would require ``continuous interference with people's
lives'' (Nozick, 1974, p.~163).

Nozick's approach represents a radical departure from egalitarian
theories in that it does not evaluate justice based on the resulting
distribution of goods or opportunities, but rather on whether that
distribution arose through processes that respect individual rights to
property and free exchange. This procedural focus offers a distinctive
perspective that has been largely absent from discussions of algorithmic
fairness.

\paragraph{2.2.3 Critiques and
Counterarguments}\label{critiques-and-counterarguments}

Both egalitarian and libertarian theories have been subject to extensive
critique. Key criticisms of egalitarian theories include concerns about
the restriction of liberty, problems with determining the appropriate
``currency'' of equality (resources, welfare, capabilities, etc.), and
questions about the moral relevance of equality itself as opposed to
sufficiency or priority (Frankfurt, 1987; Parfit, 1997).

Critiques of Nozick's entitlement theory have been equally robust. Major
objections include:

\textbf{The Problem of Initial Acquisition}: Critics argue that Nozick's
account of just acquisition is underdeveloped and that the Lockean
proviso may be impossible to satisfy in a world of finite resources
(Cohen, 1995; Otsuka, 2003).

\textbf{Historical Injustice}: Given the extensive history of injustice
in acquisition and transfer (including colonialism, slavery, and theft),
critics argue that almost all current holdings are tainted by past
injustice, rendering Nozick's theory practically inapplicable without
massive redistribution (Kymlicka, 2002; Waldron, 1992).

\textbf{Neglect of Need}: Critics contend that Nozick's theory fails to
account for basic human needs and allows for distributions that leave
some individuals in desperate poverty while others have vast wealth
(Cohen, 1995).

\textbf{Self-Ownership Without Substance}: Critics like G.A. Cohen
(1995) argue that Nozick's emphasis on self-ownership is hollow if
individuals lack access to external resources needed to make meaningful
use of their talents and abilities.

Despite these critiques, Nozick's theory remains influential and offers
valuable insights into procedural aspects of justice that are often
neglected in egalitarian approaches. The tension between procedural and
distributive conceptions of justice---between process and pattern---is
central to many debates in political philosophy and, as this thesis
argues, equally relevant to discussions of algorithmic fairness.

\subsubsection{2.3 The Intersection of Algorithmic Fairness and
Distributive
Justice}\label{the-intersection-of-algorithmic-fairness-and-distributive-justice}

\paragraph{2.3.1 Current Perspectives}\label{current-perspectives}

Recent scholarship has begun to explore the connections between
algorithmic fairness and theories of distributive justice, primarily
focusing on how different fairness metrics align with different
philosophical conceptions of fairness (Binns, 2018; Fazelpour \& Lipton,
2020; Lee \& Floridi, 2021).

This work has revealed that many common fairness metrics implicitly
adopt egalitarian perspectives. For example:

\begin{itemize}
\tightlist
\item
  Demographic parity aligns with a strict egalitarian view that
  different groups should receive equal outcomes regardless of other
  factors.
\item
  Equal opportunity metrics reflect Rawlsian concerns with fair equality
  of opportunity.
\item
  Calibration and predictive parity metrics can be linked to
  meritocratic conceptions of desert-based justice.
\end{itemize}

These connections have helped clarify the normative assumptions
underlying different technical approaches to fairness and have
highlighted the need for contextual judgment in selecting appropriate
metrics for specific applications (Lee \& Floridi, 2021).

However, as noted in the introduction, libertarian
perspectives---particularly Nozick's entitlement theory---have been
largely absent from these discussions. This absence is notable given the
significant influence of libertarian thinking in debates about
technology, markets, and regulation more broadly (Thierer, 2016).

\paragraph{2.3.2 The Gap: Libertarian Perspectives on Algorithmic
Fairness}\label{the-gap-libertarian-perspectives-on-algorithmic-fairness}

The omission of libertarian perspectives from discussions of algorithmic
fairness represents a significant gap in the literature. This gap may
stem from several factors:

\begin{enumerate}
\def\labelenumi{\arabic{enumi}.}
\item
  The apparent incompatibility between libertarian emphasis on
  individual rights and process versus the focus on group-level
  statistics in most fairness metrics.
\item
  The challenge of translating historical and procedural conceptions of
  justice into formal, quantifiable measures applicable to algorithms.
\item
  The dominant egalitarian orientation of much research on
  discrimination and bias, reflecting broader trends in social justice
  scholarship.
\item
  The difficulty of applying Nozick's principles to algorithmic contexts
  where the concepts of ``acquisition'' and ``transfer'' do not have
  obvious analogues.
\end{enumerate}

Despite these challenges, a Nozickian perspective on algorithmic
fairness offers potential insights that complement existing approaches.
By focusing on the processes by which algorithmic decisions are made,
rather than solely on the resulting distributions, such a perspective
might address some of the procedural concerns that have been raised
about current fairness metrics (Grgić-Hlača et al., 2018; Binns, 2018).
Additionally, a libertarian approach might help navigate tensions
between group fairness and individual treatment that have proven
challenging for egalitarian metrics (Binns, 2020).

In the following section, I develop a framework for algorithmic fairness
grounded in Nozick's entitlement theory, addressing the challenges of
translating his historical and procedural conception of justice into the
context of algorithmic decision-making.

\subsection{3. A Nozickian Framework for Algorithmic
Fairness}\label{a-nozickian-framework-for-algorithmic-fairness}

\subsubsection{3.1 Conceptual Foundations}\label{conceptual-foundations}

\paragraph{3.1.1 Core Elements of Nozick's Theory Relevant to
Algorithms}\label{core-elements-of-nozicks-theory-relevant-to-algorithms}

To develop a Nozickian framework for algorithmic fairness, we must first
identify which elements of Nozick's entitlement theory are most relevant
to algorithmic decision-making contexts. While the direct translation of
concepts like ``acquisition'' and ``transfer'' from physical property to
algorithmic decisions is not straightforward, several core aspects of
Nozick's theory provide valuable starting points:

\textbf{Procedural Justice}: For Nozick, justice is fundamentally about
processes rather than patterns. A just distribution is one that arises
through just processes, regardless of the resulting pattern. This focus
on procedural justice can be applied to algorithmic contexts by
emphasizing the procedures through which algorithms make decisions,
rather than solely the distribution of outcomes across groups.

\textbf{Historical Dimension}: Nozick's theory is historical---the
justice of current holdings depends on the history of how they came to
be. In algorithmic contexts, this suggests attention to the historical
processes that generate the data used by algorithms and the historical
context in which algorithmic systems operate.

\textbf{Rights and Consent}: Central to Nozick's theory is respect for
individual rights, particularly property rights and the right to engage
in voluntary exchanges. In algorithmic contexts, this might translate to
concerns about consent, data ownership, and individuals' rights
regarding how their information is used in decision-making processes.

\textbf{Non-Interference}: Nozick argues against ``continuous
interference'' to maintain patterns of distribution. In algorithmic
contexts, this might suggest skepticism toward extensive interventions
in algorithm design solely to achieve particular distributive outcomes,
especially if these interventions might violate individual rights or
distort market processes.

\textbf{Rectification}: Nozick acknowledges the need to rectify past
injustices in acquisition and transfer. In algorithmic contexts, this
principle might apply to addressing historical biases in data or
correcting for past discriminatory practices that may be perpetuated by
algorithms.

These elements provide a foundation for thinking about algorithmic
fairness from a Nozickian perspective. However, translating them into
concrete principles for algorithmic design and evaluation requires
addressing several conceptual challenges.

\paragraph{3.1.2 Translating Nozickian Concepts to Algorithmic
Contexts}\label{translating-nozickian-concepts-to-algorithmic-contexts}

The translation of Nozick's principles to algorithmic contexts requires
addressing several key questions:

\textbf{What constitutes ``just acquisition'' in algorithmic
decision-making?} In Nozick's theory, just acquisition concerns how
unowned resources come to be legitimately owned. In algorithmic
contexts, this might relate to how data is collected and how algorithmic
models are developed. Just acquisition could involve obtaining data with
proper consent, ensuring transparency about how data will be used, and
developing algorithms in ways that respect the rights and autonomy of
affected individuals.

\textbf{What constitutes ``just transfer'' in algorithmic decisions?}
For Nozick, just transfer involves voluntary exchanges free from
coercion or fraud. In algorithmic contexts, this might relate to how
decisions are made and communicated. Just transfer could involve
ensuring that algorithmic decisions are transparent, that individuals
understand the basis for decisions affecting them, and that they have
meaningful opportunities to contest or appeal decisions they believe are
unjust.

\textbf{How does the principle of rectification apply to algorithms?}
Nozick's principle of rectification addresses how to correct past
injustices in acquisition or transfer. In algorithmic contexts, this
might involve correcting for historical biases in data, addressing past
discriminatory practices that may be perpetuated by algorithms, or
providing remedies for individuals who have been harmed by unjust
algorithmic decisions.

\textbf{What rights do individuals have in relation to algorithmic
systems?} Nozick's theory is fundamentally concerned with individual
rights, particularly property rights. In algorithmic contexts, relevant
rights might include data ownership rights, rights to consent to how
one's data is used, rights to explanation or contestation of algorithmic
decisions, and rights to compensation for harms caused by algorithmic
systems.

\textbf{How do we balance procedural justice with concerns about
discriminatory outcomes?} While Nozick's theory focuses on procedural
justice rather than distributive patterns, there may be cases where
seemingly just procedures lead to outcomes that appear discriminatory.
Addressing this tension requires careful consideration of how procedural
and distributive concerns can be balanced within a broadly libertarian
framework.

Addressing these questions will help us develop concrete principles for
evaluating algorithmic fairness from a Nozickian perspective. In the
following sections, I propose such principles and develop formal metrics
based on them.

\subsubsection{3.2 Principles of Nozickian Algorithmic
Fairness}\label{principles-of-nozickian-algorithmic-fairness}

Building on the conceptual foundations outlined above, I propose the
following principles for evaluating algorithmic fairness from a
Nozickian perspective:

\paragraph{3.2.1 Principle of Just Data
Acquisition}\label{principle-of-just-data-acquisition}

An algorithm satisfies the principle of just data acquisition if:

\begin{enumerate}
\def\labelenumi{\arabic{enumi}.}
\tightlist
\item
  All data used to train and validate the algorithm was collected with
  the informed consent of the individuals to whom the data pertains, or
  from legitimately public sources.
\item
  Individuals retained meaningful control over their data, including
  rights to access, correct, delete, or restrict the use of their
  personal information.
\item
  The collection and use of data respected relevant privacy rights and
  did not involve deception or coercion.
\item
  The Lockean proviso is satisfied: the data collection did not deprive
  others of essential information resources or create significant
  informational asymmetries that undermine individual autonomy.
\end{enumerate}

This principle translates Nozick's concern with just initial acquisition
to the context of data collection and algorithm development. It
emphasizes consent, control, and respect for individual rights over
personal information.

\paragraph{3.2.2 Principle of Just Algorithmic
Processing}\label{principle-of-just-algorithmic-processing}

An algorithm satisfies the principle of just algorithmic processing if:

\begin{enumerate}
\def\labelenumi{\arabic{enumi}.}
\tightlist
\item
  The algorithm makes decisions based on factors that individuals have
  meaningfully influenced through their voluntary choices and actions,
  rather than immutable characteristics or circumstances beyond their
  control.
\item
  The algorithm's decision-making process is transparent and
  intelligible to affected individuals, allowing them to understand how
  their actions and choices influence decisions about them.
\item
  The algorithm does not violate individuals' rights or entitlements
  established through just processes (e.g., legal contracts, earned
  credentials, legitimate expectations based on past performance).
\item
  The algorithm's operation does not involve fraud, deception, or other
  forms of involuntary transfer (e.g., making decisions based on
  criteria different from those publicly claimed).
\end{enumerate}

This principle applies Nozick's focus on voluntary transfers and respect
for established entitlements to the context of algorithmic
decision-making. It emphasizes transparency, agency, and respect for
rightfully acquired claims.

\paragraph{3.2.3 Principle of Algorithmic
Rectification}\label{principle-of-algorithmic-rectification}

An algorithm satisfies the principle of algorithmic rectification if:

\begin{enumerate}
\def\labelenumi{\arabic{enumi}.}
\tightlist
\item
  It includes mechanisms to identify and correct for past injustices
  that may be perpetuated through algorithmic decisions, particularly
  those involving violations of just acquisition or transfer.
\item
  It provides meaningful opportunities for individuals to contest
  decisions they believe are unjust and to seek appropriate remedies.
\item
  It includes processes for updating and improving the algorithm in
  response to identified instances of injustice or rights violations.
\item
  When historical data reflects past discriminatory practices or unjust
  social arrangements, the algorithm incorporates appropriate
  adjustments to prevent the perpetuation of these injustices.
\end{enumerate}

This principle applies Nozick's principle of rectification to
algorithmic contexts, acknowledging that historical injustices may
require correction to achieve a just state of affairs. It recognizes
that while procedural justice is paramount, there may be cases where
past procedural violations necessitate interventions to restore justice.

\paragraph{3.2.4 Meta-Principle: Minimal Interference with Voluntary
Exchanges}\label{meta-principle-minimal-interference-with-voluntary-exchanges}

In addition to the three substantive principles outlined above, a
Nozickian approach to algorithmic fairness would include a
meta-principle regarding the role of regulation and intervention:

Interventions in algorithmic design and operation should be limited to
those necessary to ensure just acquisition, processing, and
rectification, and should minimize interference with voluntary exchanges
and individual rights. This principle reflects Nozick's concern with
``continuous interference'' to maintain particular distributive
patterns.

This meta-principle does not preclude all regulation or
intervention---Nozick himself acknowledges the legitimacy of state
action to protect rights and enforce just acquisition and transfer.
However, it suggests skepticism toward extensive interventions solely to
achieve particular distributional outcomes, especially if these
interventions might infringe on individual rights or distort market
processes.

\subsubsection{3.3 Formalizing Nozickian Fairness
Metrics}\label{formalizing-nozickian-fairness-metrics}

Translating the principles outlined above into formal metrics that can
be applied to algorithmic systems presents a significant challenge.
Unlike traditional fairness metrics, which typically focus on
statistical properties of an algorithm's outputs, Nozickian metrics must
capture procedural aspects of justice that are not easily quantified.
Nevertheless, I propose the following formal approaches to measuring
algorithmic fairness from a Nozickian perspective:

\paragraph{3.3.1 Procedural Entitlement Fairness
(PEF)}\label{procedural-entitlement-fairness-pef}

I propose a metric called Procedural Entitlement Fairness (PEF) that
evaluates the extent to which an algorithm's decisions respect
individuals' rightfully acquired entitlements. Let's define:

\begin{itemize}
\tightlist
\item
  \(E_i\) = the set of rightfully acquired entitlements of individual
  \(i\)
\item
  \(D_i\) = the decision made by the algorithm regarding individual
  \(i\)
\item
  \(R(E_i, D_i)\) = a function that measures the extent to which
  decision \(D_i\) respects the entitlements \(E_i\)
\end{itemize}

Then, Procedural Entitlement Fairness (PEF) for a population of \(n\)
individuals can be defined as:

\[PEF = \frac{1}{n} \sum_{i=1}^{n} R(E_i, D_i)\]

Where \(R(E_i, D_i)\) is normalized to a value between 0 and 1, with 1
indicating full respect for entitlements and 0 indicating complete
violation.

The challenge in implementing this metric lies in defining \(E_i\) and
\(R(E_i, D_i)\) for specific contexts. In a lending context, for
example, \(E_i\) might include an individual's credit history, repayment
record, and legitimate expectations based on past financial behavior,
while \(R(E_i, D_i)\) might measure how well the lending decision aligns
with these established entitlements.

\paragraph{3.3.2 Consent and Transparency Index
(CTI)}\label{consent-and-transparency-index-cti}

The Consent and Transparency Index (CTI) measures the extent to which an
algorithmic system respects principles of just acquisition and
transparent processing. Let's define:

\begin{itemize}
\tightlist
\item
  \(C_i\) = a binary indicator of whether individual \(i\)'s data was
  collected with proper consent (1) or not (0)
\item
  \(T_i\) = a measure of the transparency of the algorithm's decision
  process for individual \(i\), normalized to a value between 0 and 1
\item
  \(w_c\) and \(w_t\) = weights assigned to the consent and transparency
  components, respectively, with \(w_c + w_t = 1\)
\end{itemize}

Then, the Consent and Transparency Index (CTI) can be defined as:

\[CTI = \frac{1}{n} \sum_{i=1}^{n} (w_c \cdot C_i + w_t \cdot T_i)\]

This metric captures key aspects of just acquisition (consent) and just
processing (transparency), reflecting the procedural focus of Nozick's
theory.

\paragraph{3.3.3 Rectification Responsiveness Measure
(RRM)}\label{rectification-responsiveness-measure-rrm}

The Rectification Responsiveness Measure (RRM) evaluates an algorithm's
capacity to identify and correct for past injustices that may be
perpetuated through its decisions. Let's define:

\begin{itemize}
\tightlist
\item
  \(I\) = the set of identified instances where the algorithm's
  decisions may perpetuate past injustices
\item
  \(C_j\) = a measure of the adequacy of the correction implemented for
  instance \(j \in I\), normalized to a value between 0 and 1
\item
  \(A\) = a measure of the algorithm's accessibility to contestation and
  appeal, normalized to a value between 0 and 1
\item
  \(w_c\) and \(w_a\) = weights assigned to the correction and
  accessibility components, respectively, with \(w_c + w_a = 1\)
\end{itemize}

Then, the Rectification Responsiveness Measure (RRM) can be defined as:

\(RRM = w_c \cdot \left(\frac{1}{|I|} \sum_{j \in I} C_j\right) + w_a \cdot A\)

This metric captures the algorithm's ability to identify and correct for
past injustices, as well as its accessibility to contestation by
affected individuals, reflecting Nozick's principle of rectification.

\paragraph{3.3.4 Choice Sensitivity Ratio
(CSR)}\label{choice-sensitivity-ratio-csr}

The Choice Sensitivity Ratio (CSR) measures the extent to which an
algorithm's decisions are sensitive to factors that individuals can
influence through their voluntary choices, as opposed to immutable
characteristics or circumstances beyond their control. Let's define:

\begin{itemize}
\tightlist
\item
  \(F_v\) = the set of features used by the algorithm that reflect
  voluntary choices
\item
  \(F_i\) = the set of features used by the algorithm that reflect
  immutable characteristics or circumstances beyond individual control
\item
  \(I_v\) = the importance of voluntary features in the algorithm's
  decisions, measured by their collective influence on the algorithm's
  output
\item
  \(I_i\) = the importance of immutable features in the algorithm's
  decisions, measured similarly
\end{itemize}

Then, the Choice Sensitivity Ratio (CSR) can be defined as:

\(CSR = \frac{I_v}{I_v + I_i}\)

This ratio ranges from 0 to 1, with higher values indicating greater
sensitivity to voluntary choices, which aligns with Nozick's emphasis on
individual agency and responsibility.

\subsubsection{3.4 Integrated Nozickian Fairness
Framework}\label{integrated-nozickian-fairness-framework}

While the individual metrics proposed above capture specific aspects of
Nozickian fairness, an integrated framework is needed to evaluate
algorithmic systems comprehensively. I propose an integrated Nozickian
Fairness Score (NFS) that combines the four metrics:

\(NFS = w_{PEF} \cdot PEF + w_{CTI} \cdot CTI + w_{RRM} \cdot RRM + w_{CSR} \cdot CSR\)

Where \(w_{PEF}\), \(w_{CTI}\), \(w_{RRM}\), and \(w_{CSR}\) are weights
assigned to each component, with
\(w_{PEF} + w_{CTI} + w_{RRM} + w_{CSR} = 1\).

The weights can be adjusted based on the specific context and the
relative importance of different aspects of Nozickian fairness in that
context. For example, in contexts where historical injustices are
particularly salient, greater weight might be given to RRM, while in
contexts where individual agency is paramount, CSR might receive greater
weight.

This integrated framework allows for a comprehensive evaluation of
algorithmic systems from a Nozickian perspective, focusing on procedural
justice rather than distributive patterns. Importantly, it does not
preclude the possibility of unequal outcomes across demographic
groups---if these inequalities arise through just processes, they are
not considered unjust from a Nozickian perspective.

\paragraph{3.4.1 Comparison with Traditional Fairness
Metrics}\label{comparison-with-traditional-fairness-metrics}

The Nozickian fairness framework differs significantly from traditional
fairness metrics in several key respects:

\textbf{Focus on Process vs.~Outcome}: Traditional metrics like
demographic parity and equalized odds focus on the distribution of
outcomes across demographic groups. In contrast, the Nozickian framework
focuses on the processes by which decisions are made, emphasizing
consent, transparency, respect for entitlements, and sensitivity to
voluntary choices.

\textbf{Individual vs.~Group Level}: Traditional metrics typically
operate at the group level, comparing outcomes across demographic
categories. The Nozickian framework primarily operates at the individual
level, evaluating how algorithmic decisions respect each individual's
rights and entitlements.

\textbf{Historical vs.~Snapshot View}: Traditional metrics typically
take a snapshot view, evaluating the current distribution of outcomes
without reference to how that distribution came about. The Nozickian
framework takes a historical view, considering how data was acquired and
how past injustices might require rectification.

\textbf{Acceptance of Unequal Outcomes}: Traditional metrics often aim
for some form of equality or parity in outcomes across groups. The
Nozickian framework accepts potentially unequal outcomes if they arise
through just processes of acquisition, transfer, and rectification.

These differences illustrate the distinct perspective that a Nozickian
approach brings to discussions of algorithmic fairness. Rather than
asking whether an algorithm produces equal outcomes across groups, it
asks whether the algorithm respects individual rights, operates
transparently, and provides appropriate mechanisms for rectification
when needed.

\subsection{4. Application to Case
Studies}\label{application-to-case-studies}

To illustrate how the Nozickian framework for algorithmic fairness might
be applied in practice, I examine two hypothetical case studies:
algorithmic lending decisions and automated hiring systems. These
applications demonstrate the framework's practical utility while
highlighting how it differs from traditional fairness approaches.

\subsubsection{4.1 Case Study 1: Algorithmic Lending
Decisions}\label{case-study-1-algorithmic-lending-decisions}

\paragraph{4.1.1 Context}\label{context}

Consider a financial institution that uses an algorithmic system to
evaluate loan applications. The system uses various data
points---including credit history, income, employment stability, and
debt-to-income ratio---to predict an applicant's likelihood of repaying
a loan. Based on this prediction, the algorithm either approves or
denies the loan application, or offers different interest rates to
different applicants.

Traditional fairness metrics might focus on whether the algorithm's
decisions result in equal approval rates or similar interest rates
across demographic groups such as race or gender. From an egalitarian
perspective, substantial disparities in outcomes might be considered
evidence of unfairness, even if the algorithm accurately predicts
repayment likelihood.

Let us examine this scenario through the lens of Nozickian fairness,
applying the metrics developed in Section 3.

\paragraph{4.1.2 Just Data Acquisition}\label{just-data-acquisition}

To evaluate the principle of just data acquisition, we would consider:

\begin{itemize}
\tightlist
\item
  Was the data used to train and validate the algorithm collected with
  informed consent?
\item
  Do individuals retain control over their financial data, with rights
  to access, correct, and restrict its use?
\item
  Does the data collection respect privacy rights and avoid deception or
  coercion?
\item
  Does the data collection create significant informational asymmetries
  that undermine individual autonomy?
\end{itemize}

If the algorithm uses credit bureau data, we would examine whether
individuals provided informed consent for this use and whether they have
meaningful control over their credit information. If the algorithm
scrapes social media or other personal data without explicit consent,
this would violate the principle of just acquisition.

Applying the Consent and Transparency Index (CTI), we might find that
while traditional credit data is collected with some form of consent
(though perhaps not fully informed consent), the use of alternative data
sources might lack proper consent, resulting in a lower CTI score.

\paragraph{4.1.3 Just Algorithmic
Processing}\label{just-algorithmic-processing}

To evaluate the principle of just algorithmic processing, we would
consider:

\begin{itemize}
\tightlist
\item
  Does the algorithm base decisions primarily on factors that
  individuals have meaningfully influenced through their voluntary
  choices (e.g., payment history, debt management) rather than immutable
  characteristics or circumstances beyond their control (e.g., race,
  family background)?
\item
  Is the algorithm's decision-making process transparent and
  intelligible to loan applicants?
\item
  Does the algorithm respect legitimate expectations based on an
  individual's credit history and financial behavior?
\item
  Does the algorithm operate as described, without hidden factors or
  deceptive practices?
\end{itemize}

Applying the Procedural Entitlement Fairness (PEF) metric, we would
assess whether individuals with similar credit histories and financial
behaviors receive similar treatment, regardless of demographic
characteristics. The Choice Sensitivity Ratio (CSR) would measure the
extent to which loan decisions are based on voluntary financial
behaviors rather than immutable characteristics or circumstances beyond
individual control.

\paragraph{4.1.4 Algorithmic
Rectification}\label{algorithmic-rectification}

To evaluate the principle of algorithmic rectification, we would
consider:

\begin{itemize}
\tightlist
\item
  Does the algorithm account for historical discrimination in lending
  that may be reflected in credit histories?
\item
  Are there meaningful opportunities for individuals to contest loan
  denials or unfavorable terms?
\item
  Does the algorithm improve over time based on identified instances of
  unjust decisions?
\item
  Are there mechanisms to address cases where historical data reflects
  past discriminatory practices?
\end{itemize}

Applying the Rectification Responsiveness Measure (RRM), we would assess
whether the lending algorithm includes mechanisms to identify and
correct for past discriminatory lending practices that might be
perpetuated through algorithmic decisions.

\paragraph{4.1.5 Integrated Evaluation}\label{integrated-evaluation}

Combining these assessments into the integrated Nozickian Fairness Score
(NFS), we might find that a lending algorithm scores highly on some
dimensions (e.g., basing decisions on voluntary financial behaviors) but
poorly on others (e.g., limited transparency or weak rectification
mechanisms).

Importantly, from a Nozickian perspective, disparities in loan approval
rates across demographic groups would not necessarily indicate
unfairness if these disparities arose through just processes. If
individuals from different groups have different credit histories due to
their voluntary financial choices, and the algorithm bases decisions on
these histories in a transparent and consistent manner, the resulting
disparities would not be considered unjust.

However, if disparities arise from historical injustices in acquisition
or transfer (e.g., past discriminatory lending practices that affected
credit histories), the principle of rectification would require
appropriate adjustments to prevent the perpetuation of these injustices.
This might involve specific interventions to account for the effects of
past discrimination while still respecting individual financial
behavior.

\paragraph{4.1.6 Comparison with Egalitarian
Approaches}\label{comparison-with-egalitarian-approaches}

The Nozickian approach to evaluating the lending algorithm differs
markedly from traditional egalitarian approaches. While an egalitarian
approach might focus on achieving similar approval rates or loan terms
across demographic groups, the Nozickian approach focuses on the
processes by which lending decisions are made.

This difference is particularly evident in cases where disparities in
outcomes result from differences in credit histories that reflect
voluntary financial choices. An egalitarian approach might view such
disparities as problematic and advocate for interventions to achieve
more equal outcomes. A Nozickian approach would accept these disparities
as just if they arose through processes that respected principles of
just acquisition, transfer, and rectification.

However, in cases where disparities reflect historical injustices rather
than voluntary choices, both approaches might support interventions,
though for different reasons. An egalitarian approach would seek to
reduce disparities for their own sake, while a Nozickian approach would
seek to rectify past violations of just acquisition or transfer.

This comparison highlights both the distinctions and potential points of
convergence between egalitarian and libertarian approaches to
algorithmic fairness. While they differ in their fundamental
orientations---pattern versus process---they may sometimes support
similar interventions, particularly in cases involving historical
injustice.

\subsubsection{4.2 Case Study 2: Automated Hiring
Systems}\label{case-study-2-automated-hiring-systems}

\paragraph{4.2.1 Context}\label{context-1}

Consider a company that uses an automated system to screen job
applicants. The system analyzes resumes, cover letters, and possibly
video interviews to predict which candidates are likely to succeed in
the role. Based on these predictions, the system either advances
candidates to the next stage of the hiring process or rejects their
applications.

Traditional fairness metrics might focus on whether the algorithm's
decisions result in similar selection rates across demographic groups,
with disparities potentially viewed as evidence of discrimination. Let
us examine this scenario through the lens of Nozickian fairness.

\paragraph{4.2.2 Just Data Acquisition}\label{just-data-acquisition-1}

To evaluate the principle of just data acquisition, we would consider:

\begin{itemize}
\tightlist
\item
  Was the training data (e.g., past hiring decisions, employee
  performance reviews) collected with informed consent from the
  individuals involved?
\item
  Do job applicants consent to having their applications processed by an
  automated system?
\item
  Is the data collection process transparent about how applicant
  information will be used?
\item
  Does the collection and use of applicant data respect privacy rights
  and avoid deception?
\end{itemize}

Applying the Consent and Transparency Index (CTI), we would assess the
extent to which the system operates with proper consent and
transparency. If the system uses social media data or other personal
information without explicit consent, this would lower the CTI score.

\paragraph{4.2.3 Just Algorithmic
Processing}\label{just-algorithmic-processing-1}

To evaluate the principle of just algorithmic processing, we would
consider:

\begin{itemize}
\tightlist
\item
  Does the algorithm base decisions primarily on factors that reflect
  candidates' voluntary choices and actions (e.g., education, skills
  development, work experience) rather than immutable characteristics or
  circumstances beyond their control (e.g., race, gender, socioeconomic
  background)?
\item
  Is the algorithm's decision-making process transparent and
  intelligible to job applicants?
\item
  Does the algorithm respect legitimate qualifications and credentials
  that candidates have rightfully acquired?
\item
  Does the algorithm operate as described, without hidden factors or
  deceptive practices?
\end{itemize}

Applying the Choice Sensitivity Ratio (CSR), we would measure the extent
to which the hiring algorithm's decisions are sensitive to factors that
reflect candidates' voluntary choices rather than immutable
characteristics. The Procedural Entitlement Fairness (PEF) would assess
whether candidates with similar qualifications receive similar
treatment, regardless of demographic characteristics.

\paragraph{4.2.4 Algorithmic
Rectification}\label{algorithmic-rectification-1}

To evaluate the principle of algorithmic rectification, we would
consider:

\begin{itemize}
\tightlist
\item
  Does the algorithm account for historical discrimination in hiring
  that may be reflected in the training data?
\item
  Are there meaningful opportunities for candidates to contest
  rejections they believe are unjust?
\item
  Does the system improve over time based on identified instances of
  unfair decisions?
\item
  Are there mechanisms to address cases where historical hiring data
  reflects past discriminatory practices?
\end{itemize}

Applying the Rectification Responsiveness Measure (RRM), we would assess
whether the hiring algorithm includes mechanisms to identify and correct
for past discriminatory hiring practices that might be perpetuated
through algorithmic decisions.

\paragraph{4.2.5 Integrated Evaluation}\label{integrated-evaluation-1}

Combining these assessments into the integrated Nozickian Fairness Score
(NFS), we might find that a hiring algorithm scores highly on some
dimensions (e.g., respecting candidates' rightfully acquired
credentials) but poorly on others (e.g., limited mechanisms for
contestation or rectification).

From a Nozickian perspective, disparities in selection rates across
demographic groups would not necessarily indicate unfairness if these
disparities arose through just processes. If individuals from different
groups have different qualifications due to their voluntary choices, and
the algorithm bases decisions on these qualifications in a transparent
and consistent manner, the resulting disparities would not be considered
unjust.

However, if disparities arise from historical injustices that affected
educational or career opportunities, the principle of rectification
would require appropriate adjustments. This might involve specific
interventions to account for the effects of past discrimination while
still respecting individual qualifications and achievements.

\paragraph{4.2.6 Comparison with Egalitarian
Approaches}\label{comparison-with-egalitarian-approaches-1}

The Nozickian approach to evaluating the hiring algorithm differs from
traditional egalitarian approaches in its focus on process rather than
outcome patterns. While an egalitarian approach might focus on achieving
similar selection rates across demographic groups, the Nozickian
approach focuses on the processes by which hiring decisions are made.

This difference is particularly evident in cases where disparities in
outcomes result from differences in qualifications that reflect
voluntary educational and career choices. An egalitarian approach might
view such disparities as problematic and advocate for interventions to
achieve more equal outcomes. A Nozickian approach would accept these
disparities as just if they arose through processes that respected
principles of just acquisition, transfer, and rectification.

However, in cases where disparities reflect historical injustices rather
than voluntary choices, both approaches might support interventions,
though for different reasons. An egalitarian approach would seek to
reduce disparities for their own sake, while a Nozickian approach would
seek to rectify past violations of just acquisition or transfer.

This comparison highlights how different philosophical perspectives lead
to different evaluations of algorithmic fairness, with potentially
significant implications for how automated systems are designed,
deployed, and regulated.

\subsection{5. Discussion and
Implications}\label{discussion-and-implications}

\subsubsection{5.1 Philosophical
Implications}\label{philosophical-implications}

The Nozickian framework for algorithmic fairness developed in this
thesis has several important philosophical implications for how we
conceptualize and evaluate fairness in automated decision-making
systems.

\paragraph{5.1.1 Pluralism in Conceptions of Algorithmic
Fairness}\label{pluralism-in-conceptions-of-algorithmic-fairness}

The development of a libertarian approach to algorithmic fairness
alongside existing egalitarian approaches highlights the inherent
pluralism in conceptions of fairness. Different philosophical traditions
offer distinct lenses through which to evaluate algorithmic systems,
with different normative commitments and priorities. This pluralism
suggests that there is no single, universally applicable definition of
``fairness'' that can be encoded into algorithms.

Instead, the choice of fairness metrics and frameworks should be
recognized as inherently value-laden, reflecting particular
philosophical commitments about what constitutes just treatment. This
recognition calls for greater transparency about the normative
assumptions underlying different approaches to algorithmic fairness and
for more explicit engagement with the philosophical foundations of these
approaches.

\paragraph{5.1.2 Procedural vs.~Distributive Justice in Algorithmic
Contexts}\label{procedural-vs.-distributive-justice-in-algorithmic-contexts}

The Nozickian framework foregrounds procedural aspects of justice that
are often neglected in traditional fairness metrics. While most existing
metrics focus on the distribution of outcomes across demographic groups,
the Nozickian approach emphasizes the processes by which these outcomes
are determined, including issues of consent, transparency, respect for
entitlements, and rectification of past injustices.

This procedural focus offers a valuable complement to distributive
approaches, highlighting aspects of algorithmic systems that may be
morally significant regardless of their distributional effects. It
suggests that even algorithms that produce equal outcomes across groups
may be procedurally unjust if they violate principles of consent,
transparency, or respect for rightfully acquired entitlements.

\paragraph{5.1.3 The Role of History in Algorithmic
Justice}\label{the-role-of-history-in-algorithmic-justice}

Nozick's entitlement theory is fundamentally historical, evaluating the
justice of current holdings based on how they came to be. Similarly, the
Nozickian framework for algorithmic fairness emphasizes the historical
dimension of justice, considering how data was acquired, how past
injustices might affect current algorithmic decisions, and what
rectification might be required.

This historical perspective contrasts with the ahistorical approach of
most traditional fairness metrics, which evaluate the current
distribution of outcomes without reference to how that distribution came
about. The Nozickian framework suggests that historical context matters
for algorithmic fairness---that we cannot evaluate the justice of
algorithmic decisions solely by looking at their immediate effects
without considering the historical processes that shaped the data and
social contexts in which these algorithms operate.

\paragraph{5.1.4 Individual Rights vs.~Group
Fairness}\label{individual-rights-vs.-group-fairness}

The Nozickian framework prioritizes individual rights and entitlements
over group-level patterns of distribution. This individualist
orientation contrasts with the group-based focus of most traditional
fairness metrics, which compare outcomes across demographic categories.
The framework suggests that focusing exclusively on group-level
statistics may obscure important aspects of justice at the individual
level.

This tension between individual and group perspectives on justice
reflects broader debates in political philosophy about the proper
subjects of justice. The Nozickian framework suggests that individuals,
rather than groups, should be the primary subjects of justice in
algorithmic contexts, while acknowledging that group-level patterns may
be relevant evidence for identifying potential violations of individual
rights.

\subsubsection{5.2 Practical Implications}\label{practical-implications}

Beyond its philosophical significance, the Nozickian framework for
algorithmic fairness has several practical implications for how
algorithmic systems are designed, deployed, and regulated.

\paragraph{5.2.1 Implications for Algorithm
Design}\label{implications-for-algorithm-design}

The Nozickian framework suggests several principles that should guide
the design of algorithmic systems:

\textbf{Transparency and Explainability}: Algorithms should be designed
to be as transparent and explainable as possible, allowing individuals
to understand how decisions affecting them are made and how their
actions and choices influence these decisions.

\textbf{Consent-Based Data Practices}: The collection and use of data
for algorithmic systems should be based on informed consent, with
individuals retaining meaningful control over their personal
information.

\textbf{Respect for Entitlements}: Algorithms should respect
individuals' rightfully acquired entitlements, such as credentials,
qualifications, or legitimate expectations based on past performance.

\textbf{Contestability and Appeals}: Algorithmic systems should include
mechanisms for individuals to contest decisions they believe are unjust
and to seek appropriate remedies.

\textbf{Historical Awareness}: Algorithm designers should be aware of
historical injustices that may be reflected in training data and should
incorporate appropriate mechanisms to prevent the perpetuation of these
injustices.

These design principles differ from those that might be derived from
egalitarian approaches to fairness, which might focus more on achieving
particular distributional patterns across demographic groups. The
Nozickian principles emphasize procedural aspects of algorithmic systems
rather than their distributional effects.

\paragraph{5.2.2 Implications for Regulation and
Policy}\label{implications-for-regulation-and-policy}

The Nozickian framework also has implications for how algorithmic
systems are regulated and governed:

\textbf{Procedural Requirements}: Regulations might focus on procedural
requirements such as transparency, explainability, and contestability
rather than mandating particular distributional outcomes.

\textbf{Minimal Interference}: Consistent with Nozick's meta-principle
of minimal interference, regulations should be limited to those
necessary to protect individual rights and ensure just processes,
avoiding extensive interventions solely to achieve particular
distributional patterns.

\textbf{Rights-Based Approach}: Regulations might adopt a rights-based
approach, focusing on protecting individuals' rights to consent,
explanation, contestation, and rectification rather than imposing
specific fairness metrics.

\textbf{Context-Sensitivity}: Different contexts may call for different
regulatory approaches, depending on the specific rights and entitlements
at stake and the historical background against which algorithmic systems
operate.

These regulatory implications contrast with approaches that might
mandate specific distributive outcomes, such as requiring equal approval
rates across demographic groups. The Nozickian approach suggests a more
procedural and rights-based orientation to algorithmic governance.

\paragraph{5.2.3 Complementarity with Egalitarian
Approaches}\label{complementarity-with-egalitarian-approaches}

Despite the differences between Nozickian and egalitarian approaches to
algorithmic fairness, they need not be seen as mutually exclusive. In
many contexts, procedural and distributive concerns may complement each
other, with procedural requirements supporting more equitable
distributions and distributional patterns serving as evidence of
procedural fairness or unfairness.

For example, in contexts with histories of discrimination, significant
disparities in algorithmic outcomes across demographic groups might
serve as prima facie evidence of procedural unfairness, prompting
investigation into whether principles of just acquisition, transfer, or
rectification have been violated. Conversely, ensuring procedural
fairness through transparency, consent, and respect for entitlements
might naturally lead to more equitable distributions of outcomes.

This complementarity suggests that a comprehensive approach to
algorithmic fairness might draw on both Nozickian and egalitarian
perspectives, attending to both procedural and distributive aspects of
justice. Such a pluralistic approach would recognize the complex and
multifaceted nature of fairness and the need for multiple lenses through
which to evaluate algorithmic systems.

\subsubsection{5.3 Limitations and
Challenges}\label{limitations-and-challenges}

While the Nozickian framework offers valuable insights for algorithmic
fairness, it also faces several limitations and challenges that must be
acknowledged.

\paragraph{5.3.1 Practical Implementation
Challenges}\label{practical-implementation-challenges}

Several practical challenges arise when attempting to implement the
Nozickian metrics proposed in this thesis:

\textbf{Defining Entitlements}: The Procedural Entitlement Fairness
(PEF) metric requires defining what constitutes a ``rightfully acquired
entitlement'' in specific contexts, which may be complex and
contestable.

\textbf{Measuring Voluntary Choice}: The Choice Sensitivity Ratio (CSR)
requires distinguishing between factors that reflect voluntary choices
and those that reflect immutable characteristics or circumstances beyond
individual control, which is often difficult in practice.

\textbf{Identifying Past Injustices}: The Rectification Responsiveness
Measure (RRM) requires identifying instances where algorithmic decisions
may perpetuate past injustices, which presupposes agreement about what
constitutes historical injustice.

\textbf{Balancing Components}: The integrated Nozickian Fairness Score
(NFS) requires assigning weights to different components, which involves
value judgments about their relative importance.

These challenges highlight the need for context-specific implementation
of the Nozickian framework, with careful attention to the particular
rights, entitlements, and historical injustices relevant to each domain.

\paragraph{5.3.2 Theoretical Limitations}\label{theoretical-limitations}

Beyond practical challenges, the Nozickian framework also faces several
theoretical limitations:

\textbf{Initial Acquisition Problem}: As with Nozick's original theory,
the framework faces challenges regarding what constitutes just initial
acquisition, particularly in the context of data that may have been
collected under conditions of unequal power or information asymmetry.

\textbf{Historical Complexity}: The historical dimension of the
framework requires addressing complex questions about past injustices
and appropriate rectification, which may be difficult to resolve
definitively.

\textbf{Tension with Structural Perspectives}: The framework's focus on
individual rights and procedural justice may not adequately address
structural forms of injustice that operate without clear violations of
individual rights.

\textbf{Market Assumptions}: Like Nozick's theory, the framework may
rely on assumptions about the fairness of market processes that are
contested by critics who point to market failures, power imbalances, and
structural inequalities.

These theoretical limitations suggest the need for ongoing philosophical
reflection on the foundations of the Nozickian approach and its
application to algorithmic contexts.

\paragraph{5.3.3 Contextual Limitations}\label{contextual-limitations}

Finally, the Nozickian framework may be more applicable in some contexts
than others:

\textbf{Market Contexts}: The framework may be most applicable in market
contexts where voluntary exchanges and property rights are central, such
as lending or hiring.

\textbf{Public Services}: The framework may be less applicable in
contexts involving public services or goods to which individuals may
have claims based on need or citizenship rather than on entitlements
acquired through market exchanges.

\textbf{Fundamental Rights}: In contexts involving fundamental rights or
basic needs, egalitarian or sufficientarian approaches might take
precedence over the procedural focus of the Nozickian framework.

These contextual limitations highlight the importance of philosophical
pluralism in approaching algorithmic fairness, with different frameworks
being more or less appropriate depending on the specific context and the
values at stake.

\subsection{6. Conclusion}\label{conclusion}

\subsubsection{6.1 Summary of Key
Contributions}\label{summary-of-key-contributions}

This thesis has developed a framework for algorithmic fairness based on
Robert Nozick's entitlement theory of justice, offering a libertarian
perspective that complements the predominantly egalitarian approaches in
the existing literature. The key contributions of this work include:

\begin{enumerate}
\def\labelenumi{\arabic{enumi}.}
\item
  \textbf{Conceptual Framework}: The thesis has articulated a conceptual
  framework for understanding algorithmic fairness from a Nozickian
  perspective, translating the principles of just acquisition, transfer,
  and rectification into the context of algorithmic decision-making.
\item
  \textbf{Formal Metrics}: The thesis has proposed formal metrics for
  evaluating algorithmic fairness according to Nozickian principles,
  including Procedural Entitlement Fairness (PEF), Consent and
  Transparency Index (CTI), Rectification Responsiveness Measure (RRM),
  and Choice Sensitivity Ratio (CSR).
\item
  \textbf{Integrated Approach}: The thesis has developed an integrated
  Nozickian Fairness Score (NFS) that combines these metrics into a
  comprehensive evaluation framework, allowing for context-specific
  weighting of different components.
\item
  \textbf{Case Studies}: The thesis has applied the Nozickian framework
  to hypothetical case studies in algorithmic lending and hiring,
  demonstrating its practical utility and highlighting contrasts with
  traditional egalitarian approaches.
\item
  \textbf{Philosophical Analysis}: The thesis has examined the
  philosophical implications of adopting a Nozickian perspective on
  algorithmic fairness, including implications for conceptions of
  procedural vs.~distributive justice, individual rights vs.~group
  fairness, and the role of history in evaluating algorithmic systems.
\end{enumerate}

These contributions expand the philosophical foundations of algorithmic
fairness research, offering a novel perspective that foregrounds
procedural justice, individual rights, and historical context.

\subsubsection{6.2 Broader Implications}\label{broader-implications}

The Nozickian framework developed in this thesis has several broader
implications for the field of algorithmic fairness and ethics:

\textbf{Philosophical Pluralism}: By introducing a libertarian
perspective into a field dominated by egalitarian approaches, this
thesis highlights the importance of philosophical pluralism in
algorithmic ethics. Different philosophical traditions offer distinct
insights and values that can enrich our understanding of what
constitutes fair algorithmic treatment.

\textbf{Balancing Process and Pattern}: The framework suggests the need
to balance concerns about procedural justice (how decisions are made)
with concerns about distributive patterns (what outcomes result). Both
aspects are morally significant, and a comprehensive approach to
algorithmic fairness should attend to both.

\textbf{Historical Context}: The framework emphasizes the importance of
historical context in evaluating algorithmic systems, suggesting that we
cannot assess fairness without considering how data was acquired, how
past injustices might affect current decisions, and what rectification
might be required.

\textbf{Individual and Group Perspectives}: The framework highlights
tensions between individual and group perspectives on justice,
suggesting that both have important roles in evaluating algorithmic
systems but that they may sometimes point in different directions.

These broader implications suggest that algorithmic fairness research
should embrace greater philosophical diversity, contextual sensitivity,
and normative clarity about the values and principles that inform
different approaches to fairness.

\subsubsection{6.3 Future Research
Directions}\label{future-research-directions}

This thesis points to several promising directions for future research:

\textbf{Empirical Testing}: Future work could empirically test the
proposed Nozickian metrics on real-world algorithmic systems, exploring
their practical utility and limitations.

\textbf{Cross-Philosophical Integration}: Future research could explore
how Nozickian and egalitarian approaches to algorithmic fairness might
be integrated or balanced in specific contexts, drawing on the strengths
of each perspective.

\textbf{Contextual Refinement}: The Nozickian framework could be refined
and adapted for specific domains (e.g., healthcare, education, criminal
justice), attending to the particular rights, entitlements, and
historical injustices relevant to each context.

\textbf{Critique and Response}: Future work could engage more deeply
with critiques of Nozick's theory and explore how the framework might be
modified to address these critiques while maintaining its core insights.

\textbf{Regulatory Applications}: Research could examine how the
Nozickian framework might inform regulatory approaches to algorithmic
systems, balancing concerns about individual rights and procedural
justice with other social values.

These research directions would build on the foundation laid in this
thesis, further exploring the potential contributions of libertarian
political philosophy to algorithmic ethics and fairness.

\subsubsection{6.4 Concluding Reflections}\label{concluding-reflections}

The increasing role of algorithms in social decision-making raises
profound questions about justice, fairness, and the proper relationship
between individuals and automated systems. As we navigate these
questions, we need philosophical frameworks that can help us articulate
and balance the diverse values at stake.

This thesis has argued that Nozick's entitlement theory, despite its
relative absence from current discussions of algorithmic fairness,
offers valuable insights for this endeavor. By focusing on procedural
justice, individual rights, and historical context, a Nozickian approach
complements existing egalitarian perspectives, enriching our
understanding of what constitutes fair algorithmic treatment.

As we continue to develop and deploy algorithmic systems that affect
human lives, we should draw on the full range of philosophical
traditions to guide our ethical evaluations and technical innovations.
By incorporating diverse perspectives, including both egalitarian and
libertarian approaches, we can work toward algorithmic systems that
respect the complex and multifaceted nature of justice in the digital
age.

\subsection{References}\label{references}

Angwin, J., Larson, J., Mattu, S., \& Kirchner, L. (2016). Machine bias.
ProPublica, 23(2016), 139-159.

Arneson, R. J. (1989). Equality and equal opportunity for welfare.
Philosophical Studies, 56(1), 77-93.

Barocas, S., \& Selbst, A. D. (2016). Big data's disparate impact.
California Law Review, 104, 671-732.

Barocas, S., Hardt, M., \& Narayanan, A. (2019). Fairness and machine
learning. https://fairmlbook.org/

Binns, R. (2018). Fairness in machine learning: Lessons from political
philosophy. In Conference on Fairness, Accountability and Transparency
(pp.~149-159).

Binns, R. (2020). On the apparent conflict between individual and group
fairness. In Proceedings of the 2020 Conference on Fairness,
Accountability, and Transparency (pp.~514-524).

Bogen, M., \& Rieke, A. (2018). Help wanted: An examination of hiring
algorithms, equity, and bias. Upturn.

Chouldechova, A. (2017). Fair prediction with disparate impact: A study
of bias in recidivism prediction instruments. Big Data, 5(2), 153-163.

Cohen, G. A. (1989). On the currency of egalitarian justice. Ethics,
99(4), 906-944.

Cohen, G. A. (1995). Self-ownership, freedom, and equality. Cambridge
University Press.

Corbett-Davies, S., \& Goel, S. (2018). The measure and mismeasure of
fairness: A critical review of fair machine learning. arXiv preprint
arXiv:1808.00023.

Dwork, C., Hardt, M., Pitassi, T., Reingold, O., \& Zemel, R. (2012).
Fairness through awareness. In Proceedings of the 3rd innovations in
theoretical computer science conference (pp.~214-226).

Dworkin, R. (2000). Sovereign virtue: The theory and practice of
equality. Harvard University Press.

Eubanks, V. (2018). Automating inequality: How high-tech tools profile,
police, and punish the poor. St.~Martin's Press.

Fazelpour, S., \& Lipton, Z. C. (2020). Algorithmic fairness from a
non-ideal perspective. In Proceedings of the AAAI/ACM Conference on AI,
Ethics, and Society (pp.~57-63).

Frankfurt, H. (1987). Equality as a moral ideal. Ethics, 98(1), 21-43.

Fuster, A., Goldsmith-Pinkham, P., Ramadorai, T., \& Walther, A. (2022).
Predictably unequal? The effects of machine learning on credit markets.
The Journal of Finance, 77(1), 5-47.

Green, B. (2018). ``Fair'' risk assessments: A precarious approach for
criminal justice reform. In 5th Workshop on Fairness, Accountability,
and Transparency in Machine Learning.

Grgić-Hlača, N., Redmiles, E. M., Gummadi, K. P., \& Weller, A. (2018).
The case for process fairness in learning: Feature selection for fair
decision making. \emph{Proceedings of the 2018 AAAI/ACM Conference on
AI, Ethics, and Society}, 51--57.
https://doi.org/10.1145/3278721.3278725

Hardt, M., Price, E., \& Srebro, N. (2016). Equality of opportunity in
supervised learning. In \emph{Advances in Neural Information Processing
Systems} (pp.~3315--3323).

Kleinberg, J., Mullainathan, S., \& Raghavan, M. (2017). Inherent
trade-offs in the fair determination of risk scores. In
\emph{Proceedings of the 8th Innovations in Theoretical Computer Science
Conference} (ITCS).

Lee, M., \& Floridi, L. (2021). Algorithmic fairness in AI for social
good: A survey. \emph{Philosophy \& Technology}, 34, 1023--1052.

Obermeyer, Z., Powers, B., Vogeli, C., \& Mullainathan, S. (2019).
Dissecting racial bias in an algorithm used to manage the health of
populations. \emph{Science}, 366(6464), 447--453.

Otsuka, M. (2003). \emph{Libertarianism without inequality}. Oxford
University Press.

Parfit, D. (1997). Equality and priority. \emph{Ratio}, 10(3), 202--221.

Rawls, J. (1971). \emph{A theory of justice}. Harvard University Press.

Selbst, A. D., Boyd, D., Friedler, S. A., Venkatasubramanian, S., \&
Vertesi, J. (2019). Fairness and abstraction in sociotechnical systems.
In \emph{Proceedings of the Conference on Fairness, Accountability, and
Transparency} (pp.~59--68).

Thierer, A. (2016). Permissionless innovation: The continuing case for
comprehensive technological freedom. \emph{Mercatus Center at George
Mason University.}

Waldron, J. (1992). Superseding historic injustice. \emph{Ethics},
103(1), 4--28.

Kymlicka, W. (2002). \emph{Contemporary political philosophy: An
introduction} (2nd ed.). Oxford University Press.

Nielsen, K. (1979). Radical egalitarian justice: Justice as equality.
\emph{Social Theory and Practice}, 5(2), 209--226.
