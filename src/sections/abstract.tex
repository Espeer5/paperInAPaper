In recent years, there has been growing interest in grounding algorithmic
fairness in philosophical theories of distributive justice to clarify the
meaning of fairness metrics used in automated decision-making. Most existing
approaches focus on egalitarian theories, such as Rawlsian liberal
egalitarianism. However, it remains unclear how these frameworks extend to
Nozick’s theory of entitlement justice, which defines justice in terms of
legitimate acquisition and transfer of resources rather than outcome
distributions. This omission is notable given entitlement theory’s prominence in
debates on distributive justice and leaves a substantial gap in our
understanding of how different conceptions of justice inform algorithmic
fairness. In this paper, we address this gap by exploring the relationship
between algorithmic fairness and entitlement justice. We propose a framework
called entitlement fairness, which interprets algorithmic fairness through the
lens of entitlement theory. We illustrate how this framework can be applied to
real-world decision-making, using college admissions as a case study. Our
analysis demonstrates that entitlement fairness offers a nuanced,
context-sensitive approach to evaluating fairness in algorithms, expanding the
philosophical foundations of the field.