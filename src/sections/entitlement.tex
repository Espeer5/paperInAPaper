An entitlement theory of justice is a distributive theory of justice which
posits the following distribution rule: Allocate amount $R$ of resource $X$ to 
agent $A$ if and only if $A$ is entitled to $R$ of $X$. An entitlement in this
context is a \textit{property right} held by the agent over the resource. 
Different entitlement theories of justice differ in the criteria they use to 
determine entitlements, and the concept of property rights they endorse.
Here we will detail the entitlement theory of justice as proposed
by~\cite{Nozick_1974} and its issues, then discuss more recent efforts at 
reconciling the theory with the demands of justice.

\subsection{Nozick's Entitlement Theory of Justice}

Nozick's entitlement theory of justice, often called the concept of libertarian
justice, is a theory of justice that was developed as a fundamental challenge to
Rawl's liberal egalitarianism. On the liberal egalitarian view, ensuring justice
is an inherently redistributive task. The justice of a distribution of resources
is determined by the extent to which it is equal over individuals, and there is
an implied moral responsibility to redistribute resources to those who lack them
to increase the overall equality of the distribution. This ideology provides a
strong defense of taxation and welfare programs, which redistribute resources
in order to flatten the distribution of wealth~\cite{Rawls_1971}.

Libertarian justice takes issue with the consequences of adopting this view.
Nozick asks us to consider a thought experiment. Suppose we began with an equal
distribution of resources across society. People in this society have the
freedom to choose how to use their resources, and to exchange them with others 
as they feel is fair. Many people are willing to pay to see Wilt Chamberlain 
play basketball, and so they each pay him a small amount of money to see him 
play. Over time, Chamberlain will accumulate a large sum of money though his 
efforts. The distribution of resources in the society will no longer be equal,
but will be skewed towards Chamberlain. On the egalitarian account, this excess 
wealth that Chamberlain has accumulated is unjust, and must be taken and
redistributed across society. On the libertarian view, however, Chamberlain has
gained an entitlement to his accumulated wealth, and to take it away from him
is akin to stealing. After all, if this wealth is taken away from him, then he
will have received nothing for his efforts, and enjoyed no fruits of his labor.

In Nozick's theory, people gain entitlements over resources in accordance with
3 principles:
\begin{enumerate}
    \item The principle of justice in acquisition: A person who acquires a
          resource through a just process is entitled to that resource.
          A process of acquisition is just if the acquisition is in accordance
          with Lockean proviso (discussed below).
    \item The principle of justice in transfer: A person who acquires a resource
          through a just transfer is entitled to that resource. A transfer is 
          just if the transfer is voluntary and the resource is transferred
          from someone who is entitled to it.
    \item The principle of rectification: A person who acquires a resource
          through the rectification of a prior injustice is entitled to that
          resource. Rectification must be proportional to the injustice which
          is being rectified.
\end{enumerate}

On analysis, one will see that a key difference between this libertarian view
and the liberal egalitarian view is the fundamental unit of justice. For the 
liberal egalitarian, justice is realized in the distribution of resources
itself. This approach is refered to as a patterned or \textit{end-state} view of justice.
For the libertarian, justice is realized in the process by which resources are
acquired and transferred. This approach is refered to as a \textit{historical} theory of
justice. In order to determine if the current state of affairs is just with 
respect to a particular holding, one must trace the history of that holding
back to its original acquisition, and ensure that each step in the process was
just. For Nozick, any end-state theory of justice is inherently flawed, as it 
requires the restriction of individual liberties~\cite{Henberg_1977}. It is
plain that this view of justice hinges strongly on being able to identify and
justify the initial acquisition of resources, else the theory can say nothing 
about the justice of the current distribution of resources.

\subsection{The Justification of Acquisition}

For Nozick the Lockean proviso underscored the principle of justice in
acquisition. The proviso contains two parts. The first part is a mechanism for 
justifying the initial acquisition of resources. It begins with the inherent 
right of self-ownership that all individuals possess. Locke argued that when 
an individual mized their own labor with a resource, they transferred some of 
themself into the resource, and so extended their right of self-ownership over 
the resource, thereby obtaining an entitlement to it. The second part of the
proviso, almost as an afterthought, is a restriction on the extent to which 
resources can be acquired. It states that a person can only acquire a resource
if there is enough and as good left over for others. This restriction is
necessary to ensure that the acquisition of resources does not infringe on the
rights of others to acquire resources.

Other accounts of entitlement justice have used different mechanisms to justify
the acquisition of resources..~\cite{Mack_1990} proposed that the acquisition of
resources could be justified as a separate unalienable right that all
individuals possess.~\cite{Van_der_Veen_1985} proposed that the acquisition of
resources could be justified consequentially by the net utility that the
acquisition brings to society In general, Van Der Veen showed that given a
particular type of holding, one can specify a theory of entitlement justice with
a corresponding utilitarian theory of acquisition that can be used as a basis
for determining entitlements.

\subsection{Critiques of Entitlement}

Nozick's entitlement theory is heavily criticized on its foundation in the
Lockean proviso. The final clause of the proviso provides a restriction on the
extent to which resources can be acquired, but is a weak restriction that makes
it difficult to justify the acquisition of resources in practice. There are two 
mechanisms by which the proviso as it pertains to Nozick's entitlement theory 
breaks down.

Firstly, the proviso is a weak and vague restriction. It was written in an era 
when it seemed plausible that individuals would frequently be staking claim over
new posessions in the wilderness, in particular parcels of land. However, in the
modern setting, there are few unclaimed natural resources, and those that exist
come under heavy contention for acquisition. The proviso does not provide a
clear mechanism for dividing up the resources in this case, and it seems
entirely unlikely that one can satisfy both aspects of the proviso concurrently.

Secondly, the proviso has a problem dealing with the issue of surplus value. 
According to the proviso, when an individual acquires a resource, they acquire
it by instilling some valuable portion of themself into the resource. There is
thus a fixed amount of value transferred onto the resource through the person's
labor. However, in a free market like the one Nozick describes in his 
theory of entitlement justice, the value of a resource is not fixed, it is
dictated by market forces. If an individual acquires a resource and then the 
value of that resource increases due to scarcity or high demand, then the
individual can trade their resource and gain entitlement over property with a 
value greater than that which they instilled into their original acquisition.

These issues provide a strong challenge to Nozick's entitlement theory as they
can result in disastrous consequences. Besides the proviso, Nozick's theory may
be criticized for its potential to justify unacceptable outcomes through
transfer. For example, someone who is starving may ``voluntarily'' agree to
trade property for food whose value is far below the value of the property, and
per Nozick, this trade might be considered just.  Critically, this does not
spell the end for the entitlement theory of justice, but it does suggest that
the underlying theory of property rights for a successful entitlement theory of
justice as well as the restrictions on the types of transfers it can justify
must be more nuanced than what Nozick proposed.

\subsection{Instrumental Property Rights}

As described, ore modern theories of entitlement have sought to address these
issues by replacing the Lockean proviso with an alternate theory of property
rights.~\cite{Van_der_Veen_1985} showed that entitlement systems existed on a 
spectrum such that the theory of property rights used could be tailored to the
resource being distributed. For example, the theory of property rights used to
distribute land could be different from the theory of property rights used to
distribute food or water.

Regardless of the thoery of property rights used, to overcome the challenges of
Nozick's theory, one must prevent an entitlement theory from justifying morally
unacceptable outcomes. ~\cite{Sen_1988} shows that while the interpretation of
property rights as inerently valuable and inalienable leads to severe issues of
poverty and hunger, the interpretation of property rights as
\textit{instrumental} rights, which are valuable only insofar as they lead to
good outcomes, can be used to develop systems of entitlement without such
issues. Instrumental property rights cannot supersede the demands of basic
necessity for all agents, and so can be used to develop systems of entitlement
which protect property rights while avoiding the issues of the Lockean proviso.

Combining these lessons, we realize that a successful modern theory of 
entitlement justice is one situated atop a theory of domain specific and 
instrumental property rights. For a given type of holding or resource, the
theory of property rights must be tailored to the resource, and must be created
and enforced with the full scope of its consequences in mind. ``Re-situating''
the Nozickean entitlement theory atop such a theory of property rights will 
allow us to develop a theory of entitlement justice which can be used to
determine the justice of a distribution of resources in a modern society.
