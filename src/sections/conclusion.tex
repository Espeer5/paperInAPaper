In this paper, we have presented a novel approach to the problem of justifying
algorithmic fairness criteria. We have presented a formalism for discussion of
algorithmic fairness that is grounded in the theory of distributive justice, and
used that formalism to demonstrate the limitations of existing fairness criteria
with respect to distributive justice. In order to overcome these limitations, we
have suggested turning to the theory of entitlement justice, which has been
largely overlooked in this area of inquiry. Leveraging entitlement theory as a
political philosophy, we have proposed a new fairness criterion, the
\emph{entitlement fairness criterion}, which has a clear and structured 
connection to theories of justice derived from political philosophy. The
benefits of this approach are threefold.

First, the entitlement fairness criterion forced algorithm designers to confront
the moral reasoning inherent in their design choices. In grounding fairness in
entitlement justice, designers are made to present a clear account of property
rights specific to the problem domain in question. This is a significant
improvement over existing fairness criteria, which often rely on vague notions
of fairness that are not grounded in any particular theory of justice, and may
hide normative assumptions that are not immediately apparent.

Second, the entitlement fairness criterion provides a clear and structured
framework for evaluating the fairness of algorithms. Rather than choosing from
an ad hoc list of fairness criteria, designers can use the entitlement fairness
criterion to evaluate the fairness of their algorithms in a systematic and
consistent manner. This allows for a more rigorous evaluation of the fairness of
algorithms, and can help to identify and correct biases that may be present in
the design of algorithms.

Finally, the entitlement fairness criterion provides a principled basis for
human intervention in the decision-making process of algorithms. By grounding
fairness in property rights which are instrumental in nature, the entitlement
fairness criterion explicitly demands human oversight and correction in cases
where the algorithm fails to respect rights which take higher precedence than
property rights do. This bakes in a broader awareness and accountability for the
full range of ethical considerations that may be at play in the design and
implementation of sociotechnical systems.

In conclusion, we believe that the entitlement fairness criterion offers a
promising new approach to the problem of justifying algorithmic fairness
criteria. Future work in this area should focus on developing a more detailed
account of how to formulate property rights in specific problem domains, and on
exploring how the entitlement fairness criterion can be used to evaluate the
fairness of algorithms in production. We hope that this paper will inspire
further research in this area, and that the entitlement fairness criterion will
become a valuable tool for ensuring that algorithms are designed and deployed in
a fair and just manner.
