%% INTRO SECTION  %%

The rise of algorithmic decision making in the public sector has caused
significant public concern. Algorithms increasingly make decisions that affect
individuals' lives, from determining creditworthiness to predicting criminal 
recidivism, and the public has grown cautious of their potential to perpetuate 
existing social inequalities. A 2018 study showed that 58\% of Americans
believe that algorithms will always have some level of bias~\cite{Smith_2018}, 
and as documented in the famed COMPAS case, these fears are not
unfounded~\cite{Angwin_2016}.

In response to these concerns, a growing body of research has focused on
developing algorithmic fairness measures to evaluate and mitigate the biases
in algorithmic decision making. A large number of different measures have been
proposed~\cite{CorbettDavies_2023} and applied to a wide range of problems.
However, many questions remain unanswered about the theoretical foundations of
these measures and their relationship to broader sociotechnical systems. In
particular, the relationship of these measures to philosophically rigorous
definitions of justice is not well understood.

Most recently,~\cite{Binns_2018},~\cite{Hertweck_2024}, and~\cite{Kuppler_2021}
have explored this relationship, attempting to make sense of the theoretical
foundations of algorithmic fairness measures through the lens of distributive
theories of justice in political philosophy. A distributive theory of justice is
a normative framework that provides principles and criteria for allocating
resources, benefits, and burdens among individuals or groups within a society,
with the aim of achieving a just and fair distribution. At a first glance, it
is appealing to compare such a scheme to the problem of algorithmic fairness —
encode the constraints on the allocation into the fairness measure, and
optimize for the measure to achieve a fair allocation. However, this approach
demands further investigation. How do we encode the constraints of a
distributive theory of justice into a fairness measure? How well do the measures
in the literature capture competing accounts of distributive justice?

The papers mentioned above all argue that existing fairness metrics implement
different forms of egalitarian justice, primarily luck egalitarianism. In this
paper, we will carefully examine the relationship between algorithmic fairness
and distributive justice, and argue that the existing literature is too narrow
its scope. We will argue that algorithmic fairness should be understood through
the lens of entitlement justice, a distributive theory of justice that has been
largely overlooked in the literature. We will argue that entitlement theory,
under which justice is rooted in the idea of respecting individuals'
property rights, offers a more nuanced and context-sensitive understanding of
algorithmic fariness. We argue that by conceptualizing the algorithmic fairness
problem through the lens of entitlement justice, system designers are forced to
clearly present the inherent normative reasoning and values endorsed by their
systems.

The rest of this paper is organized as follows. In Section~\ref{sec:background},
we provide an overview of the existing literature on algorithmic fairness and
distributive justice. We draw on the formalism from~\cite{Kuppler_2021}
and~\cite{CorbettDavies_2023} to create a unified model for understanding
algorithmic fairness and distributive justice consistently with each other. 
In Section~\ref{sec:entitlement-justice}, we introduce the concept of
entitlement justice and discuss its historical development. We confront the 
traditional objections to entitlement theory and show how they can be overcome
in the context of algorithmic decision making. In
Section~\ref{sec:entitlement-fairness}, we propose a new framework for
understanding algorithmic fairness through the lens of entitlement justice. We
analyze the implications of this framework for existing algorithmic fairness
measures and show an example of how it can be applied to a real-world case
study. Finally, in Section~\ref{sec:conclusion}, we conclude with a discussion
of the broader implications of our work and suggest directions for future
research.
