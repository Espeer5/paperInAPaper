%% INTRO SECTION  %%

The rise of algorithmic decision making in the public sector has caused
significant public concern. Algorithms increasingly make decisions that affect
individuals' lives, from determining creditworthiness to predicting criminal 
recidivism, and the public has grown cautious of their potential to perpetuate
and exacerbate existing social inequalities. A 2018 study showed that 58\% of
Americans believe that algorithms will always have some level of
bias~\citep{Smith_2018}, and as documented in the famed COMPAS case, these fears
are not unfounded~\citep{Angwin_2016}.

In response to these concerns, a growing body of research has focused on
developing algorithmic fairness measures to evaluate and mitigate the biases
in algorithmic decision making. A large number of different measures have been
proposed~\citep{CorbettDavies_2023} and applied to a wide range of problems.
However, many questions remain unanswered about the theoretical foundations of
these measures and their relationship to broader sociotechnical systems. In
particular, the relationship of these measures to philosophically rigorous
definitions of justice is not well understood.

In an effort to develop the theoretical foundations of algorithmic fairness,
researchers have turned to the field of distributive justice for guidance.
A distributive theory of justice is a normative framework that provides
principles and criteria for allocating benefits and burdens among individuals or
groups within a society, with the aim of achieving a just and fair distribution.
The field can be seen as polarized along an axis from liberal egalitarianism to
entitlement theory. Under the liberal view, commonly associated with John Rawls, 
the chief objective of justice is to equalize allocation across all individuals
in a population. In contrast, the entitlement view, associated with Robert
Nozick, emphasizes the importance of individual property rights and the freedom
to exchange goods and services without interference. 

Recent papers (~\citep{Binns_2018},~\citep{Hertweck_2024}, and~\citep{Kuppler_2021})
have explored the relationship between algorithmic fairness and theories of
distributive justice. Efforts have largely focused on grounding fairness
measures in liberal justice, while theories of libertarian justice have been
largely overlooked in the literature. Given that libertarian justice is a
prominent area of inquiry in political philosophy that addresses a broad range
of concerns not covered in liberal justice, it is worth investigating how to
close this gap. In particular, how do the concerns of libertarian justice appear
in algorithmic decision making? How can these concerns be encoded by algorithmic
fairness measures? And what do we stand to lose or gain by conceptualizing
algorithmic fairness through the libertarian lens?

In this paper, we will carefully examine the relationship between algorithmic
fairness and libertarian justice, and develop a formalism that lays clear the
relationship between the two. We will demonstrate that libertarian justice can
be encoded within a measure of algorithmic fairness, and show that doing show
offers a nuanced and context-sensitive means of understanding algorithmic
fairness. We will argue that by conceptualizing algorithmic fairness through
libertarian justice, system designers are made to clearly present the inherent
normative reasoning and values embedded in their systems.

\todo{Sharped objectives in previous paragraph}

The rest of this paper is organized as follows. In Section~\ref{sec:background},
we provide an overview of the existing literature on algorithmic fairness and
distributive justice. We draw on the formalism from~\citep{Kuppler_2021}
and~\citep{CorbettDavies_2023} to create a unified model for understanding
algorithmic fairness and distributive justice consistently with each other. 
In Section~\ref{sec:entitlement-justice}, we introduce the concept of
entitlement justice and discuss its historical development. We contrast
entitlement theory with liberal egalitarianism to identify the the critical
elements of entitlement which must be represented in account of algorithmic
fairness, and confront the traditional objections to entitlement theory. In
Section~\ref{sec:entitlement-fairness}, we propose a new framework for
understanding algorithmic fairness through the lens of entitlement justice. We
analyze the implications of this framework for existing algorithmic fairness
measures and show an example of how it can be applied to a real-world case
study. Finally, in Section~\ref{sec:conclusion}, we conclude with a discussion
of the broader implications of our work and suggest directions for future
research.
